%%%%%%%%%%%%%%%%%%%%%%%%%%%%%%%%%%%%%%%%%
% Programming/Coding Assignment
% LaTeX Template
%
% Original author:
% Ted Pavlic (http://www.tedpavlic.com)
%
%
% This template uses a Perl script as an example snippet of code, most other
% languages are also usable. Configure them in the "CODE INCLUSION 
% CONFIGURATION" section.
%
%%%%%%%%%%%%%%%%%%%%%%%%%%%%%%%%%%%%%%%%%

%----------------------------------------------------------------------------------------
%	PACKAGES AND OTHER DOCUMENT CONFIGURATIONS
%----------------------------------------------------------------------------------------

\documentclass{article}
\usepackage[utf8]{inputenc}

\usepackage[swissgerman]{babel}
\usepackage{amsmath}
\usepackage{amsfonts}
\usepackage{fancyhdr} % Required for custom headers
\usepackage{lastpage} % Required to determine the last page for the footer
\usepackage{extramarks} % Required for headers and footers
\usepackage[usenames,dvipsnames]{color} % Required for custom colors
\usepackage{graphicx} % Required to insert images
\usepackage{listings} % Required for insertion of code
\usepackage{courier} % Required for the courier font
\usepackage{enumerate} % used for enumerate args
\usepackage{multicol} % columns

\usepackage{pgf} 
\usepackage{tikz}
\usepackage{forest} % treees :D
\usetikzlibrary{arrows,automata} %for FSM

% Custom commands
\DeclareMathOperator{\Kl}{Kl} %Klassen von Zuständen

\usepackage{mathtools}
\DeclarePairedDelimiter{\ceil}{\lceil}{\rceil}
% Shamelessly copied from http://tex.stackexchange.com/questions/43008/absolute-value-symbols
\DeclarePairedDelimiter\abs{\lvert}{\rvert} % nice |x|
\DeclarePairedDelimiter\norm{\lVert}{\rVert} % nice ||x||
% Swap the definition of \abs* and \norm*, so that \abs
% and \norm resizes the size of the brackets, and the 
% starred version does not.
\makeatletter
\let\oldabs\abs
\def\abs{\@ifstar{\oldabs}{\oldabs*}}
\let\oldnorm\norm
\def\norm{\@ifstar{\oldnorm}{\oldnorm*}}
\makeatother


% Margins
\topmargin=-0.45in
\evensidemargin=0in
\oddsidemargin=0in
\textwidth=6.5in
\textheight=9.0in
\headsep=0.25in

\linespread{1.1} % Line spacing

% Set up the header and footer
\pagestyle{fancy}
\lhead{\hmwkAuthorName} % Top left header
%\chead{\hmwkClass\ (\hmwkClassInstructor\): \hmwkTitle} % Top center head
%\rhead{\firstxmark} % Top right header
\rhead{}
\lfoot{\lastxmark} % Bottom left footer
\cfoot{} % Bottom center footer
\rfoot{Seite\ \thepage\ von\ \protect\pageref{LastPage}} % Bottom right footer
\renewcommand\headrulewidth{0.4pt} % Size of the header rule
\renewcommand\footrulewidth{0.4pt} % Size of the footer rule

\setlength\parindent{0pt} % Removes all indentation from paragraphs

%----------------------------------------------------------------------------------------
%	CODE INCLUSION CONFIGURATION
%----------------------------------------------------------------------------------------

\definecolor{MyDarkGreen}{rgb}{0.0,0.4,0.0} % This is the color used for comments
\lstloadlanguages{Pascal} % Load Pascal syntax for listings, for a list of other languages supported see: ftp://ftp.tex.ac.uk/tex-archive/macros/latex/contrib/listings/listings.pdf
\lstset{language=Perl, % Use Pascal in this example
        frame=single, % Single frame around code
        basicstyle=\small\ttfamily, % Use small true type font
        keywordstyle=[1]\color{Blue}\bf, % Pascal functions bold and blue
        keywordstyle=[2]\color{Purple}, % Pascal function arguments purple
        keywordstyle=[3]\color{Blue}\underbar, % Custom functions underlined and blue
        identifierstyle=, % Nothing special about identifiers                                         
        commentstyle=\usefont{T1}{pcr}{m}{sl}\color{MyDarkGreen}\small, % Comments small dark green courier font
        stringstyle=\color{Purple}, % Strings are purple
        showstringspaces=false, % Don't put marks in string spaces
        tabsize=5, % 5 spaces per tab
        %
        % Put standard Pascal functions not included in the default language here
        morekeywords={rand},
        %
        % Put Pascal function parameters here
        morekeywords=[2]{on, off, interp},
        %
        % Put user defined functions here
        morekeywords=[3]{test},
        %
        morecomment=[l][\color{Blue}]{...}, % Line continuation (...) like blue comment
        numbers=left, % Line numbers on left
        firstnumber=1, % Line numbers start with line 1
        numberstyle=\tiny\color{Blue}, % Line numbers are blue and small
        stepnumber=5 % Line numbers go in steps of 5
}

% Creates a new command to include a perl script, the first parameter is the filename of the script (without .p), the second parameter is the caption
\newcommand{\pascalscript}[2]{
\begin{itemize}
\item[]\lstinputlisting[caption=#2,label=#1]{#1.p}
\end{itemize}
}

%----------------------------------------------------------------------------------------
%	DOCUMENT STRUCTURE COMMANDS
%	Skip this unless you know what you're doing
%----------------------------------------------------------------------------------------

% Header and footer for when a page split occurs within a problem environment
%\newcommand{\enterProblemHeader}[1]{
%\nobreak\extramarks{#1}{#1 continued on next page\ldots}\nobreak
%\nobreak\extramarks{#1 (continued)}{#1 continued on next page\ldots}\nobreak
%}

% Header and footer for when a page split occurs between problem environments
%\newcommand{\exitProblemHeader}[1]{
%\nobreak\extramarks{#1 (continued)}{#1 continued on next page\ldots}\nobreak
%\nobreak\extramarks{#1}{}\nobreak
%}

\setcounter{secnumdepth}{0} % Removes default section numbers
\newcounter{homeworkProblemCounter} % Creates a counter to keep track of the number of problems

\newcommand{\homeworkProblemName}{}
\newenvironment{homeworkProblem}[1][Aufgabe \arabic{homeworkProblemCounter}]{ % Makes a new environment called homeworkProblem which takes 1 argument (custom name) but the default is "Problem #"
\stepcounter{homeworkProblemCounter} % Increase counter for number of problems
\renewcommand{\homeworkProblemName}{#1} % Assign \homeworkProblemName the name of the problem
\section{\homeworkProblemName} % Make a section in the document with the custom problem count
%\enterProblemHeader{\homeworkProblemName} % Header and footer within the environment
}{
%\exitProblemHeader{\homeworkProblemName} % Header and footer after the environment
}

\newcommand{\problemAnswer}[1]{ % Defines the problem answer command with the content as the only argument
\noindent\framebox[\columnwidth][c]{\begin{minipage}{0.98\columnwidth}#1\end{minipage}} % Makes the box around the problem answer and puts the content inside
}

\newcommand{\homeworkSectionName}{}
\newenvironment{homeworkSection}[1]{ % New environment for sections within homework problems, takes 1 argument - the name of the section
\renewcommand{\homeworkSectionName}{#1} % Assign \homeworkSectionName to the name of the section from the environment argument
\subsection{\homeworkSectionName} % Make a subsection with the custom name of the subsection
%\enterProblemHeader{\homeworkProblemName\ [\homeworkSectionName]} % Header and footer within the environment
}{
%\enterProblemHeader{\homeworkProblemName} % Header and footer after the environment
}

%----------------------------------------------------------------------------------------
%	NAME AND CLASS SECTION
%----------------------------------------------------------------------------------------

\newcommand{\hmwkTitle}{Übungsaufgaben\ -\ Blatt\ 4} % Assignment title
\newcommand{\hmwkDueDate}{17.\ Oktober\ 2014} % Due date
\newcommand{\hmwkClass}{Theoretische Infomatik} % Course/class
\newcommand{\hmwkClassInstructor}{Prof. Hromkovič} % Teacher/lecturer
\newcommand{\hmwkAuthorName}{Kevin Klein, Vincent von Rotz und David Bimmler} % Your name

%----------------------------------------------------------------------------------------
%	TITLE PAGE
%----------------------------------------------------------------------------------------

\title{
\vspace{2in}
\textmd{\textbf{\hmwkClass:\ \hmwkTitle}}\\
\normalsize\vspace{0.1in}\small{Abgabe\ bis\ \hmwkDueDate}\\
\vspace{0.1in}\large{\textit{\hmwkClassInstructor}
\vspace{3in}
}}
\author{\textbf{\hmwkAuthorName}}
\date{} % Insert date here if you want it to appear below your name

%----------------------------------------------------------------------------------------

\begin{document}

\maketitle

%----------------------------------------------------------------------------------------
%	TABLE OF CONTENTS
%----------------------------------------------------------------------------------------

%\setcounter{tocdepth}{1} % Uncomment this line if you don't want subsections listed in the ToC

\addtocounter{homeworkProblemCounter}{9}
\newpage
%\tableofcontents
%\newpage

%----------------------------------------------------------------------------------------
%	PROBLEM 10
%----------------------------------------------------------------------------------------

\begin{homeworkProblem}
\begin{enumerate}[(a)]
\item De vinc isch unhappy
\item Wir führen einen Widerspruchsbeweis. Wir nehmen an, die Sprache $L_2$ ist regulär. Man betrachte $\lambda, c^{2\cdot1}, c^{2\cdot2}, c^{2\cdot3}, \dots, c^{2\cdot\abs{Q}}$. Mit dem Schubfachprinzip
\begin{align*}
&\implies \exists \; i,j \mid i \neq j \land \hat{\delta}_{A}(q_0, c^{2i}) = \hat{\delta}_{A}(q_0, c^{2j}) \quad i,j \in [1,\abs{Q}] \\
&\implies \hat{\delta}_{A}(q_0, c^{2i}a^ib^i)\in F \land \hat{\delta}_{A}(q_0, c^{2j}a^ib^i) \not\in F
\end{align*}
Dies steht jedoch im Widerspruch zu Lemma 3.3, welches für alle regulären Sprachen gelten muss.

\end{enumerate}
\end{homeworkProblem}
%----------------------------------------------------------------------------------------
%	PROBLEM 11
%----------------------------------------------------------------------------------------

\begin{homeworkProblem}
\begin{enumerate}[(a)]
\item Wir führen einen Widerspruchsbeweis. Man nimmt also an, $L_3$ sei regulär. Nun existiert nach pumping Lemma ein $n_0$, sodass es eine Zerlegung $w=yxz, \abs{w}$ für jedes Wort $w$ gibt. Diese Zerlegung muss dann die drei Bedingungen erfüllen.

Sei nun $w:=a^{n_0}b^{n_0}. w \not\in{L_3}.$ Aber für jede Wahl von x gilt $yx^iz \in{L_3}$ für $i\neq 1.$ Somit ist Folgerung 3 des pumping Lemmas nicht erreicht und wir haben einen Widerspruch. Folglich ist $L_3$ nicht regulär.
\item Wir führen einen Widerspruchsbeweis. Wir nehmen an, die Sprache $L_4$ ist regulär. Sei $p_0 \geq n_0$ eine Primzahl. Dann muss aufgrund des Pumping-Lemmas eine Zerlegung $yxz = 0^{p_0}$ existieren. Es gilt auch $\abs{x} \geq 1, \abs{yx} \leq n_0$. Folglich ist $x=0^j$.

Da $yxz = 0^{p_0} \in L$, muss auch $yx^iz=0^p, p$ ist eine Primzahl. Aber für $i=p_0 + 1$ gilt \[ yx^{p_0 + 1}z=yxx^{p_0}z = 0^{p_0}x^{p_0} = 0^{p_0}0^{jp_0} = 0^{(j+1)p_0} \]
$(j+1)p_0$ ist aber keine Primzahl und somit ist das Pumping-Lemma nicht erfüllt, und die Sprache $L_4$ nicht regulär.
\end{enumerate}
\end{homeworkProblem}

\clearpage

%----------------------------------------------------------------------------------------
%	PROBLEM 12
%----------------------------------------------------------------------------------------

\begin{homeworkProblem}
\begin{enumerate}[(a)]
\item Der folgende nichtdeterministische endliche Automat akzeptiert die Sprache \[ L = \{ x\in \{ 0,1\}^{*} \mid \abs{x}_1 \!\!\mod{3} =0 \text{ oder } x \text{ enthält das Teilwort } 101\}. \]
Die Idee des Entwurfes ist, dass sich die Spezifikation der Sprache gut aufteilen lässt, in eine Sprache welche das Teilwort 101 enthält, und eine Sprache mit $\abs{x}_1 \!\!\mod{3} =0$.
\begin{center}
\begin{tikzpicture}[->,>=stealth',shorten >=1pt,auto,node distance=2.8cm,
                    semithick]
  \tikzstyle{every state}=[fill=none,draw=black,text=black]
  
  \node[initial,state,accepting]	(A)						{$q_0$};
  \node[state]			(B) [right of=A]		{$q_1$};
  \node[state]			(C) [right of=B]		{$q_2$};
  \node[state]			(D) [below right of=A,node distance=2cm]	{$q_3$};
  \node[state]			(E) [below right of=D,node distance=2cm]	{$q_4$};
  \node[accepting,state]			(F) [below right of=E,node distance=2cm]	{$q_5$};    
    
  \path (A) edge  [loop above]	node {0}		(A)
		(A) edge  				node {1} 		(B)
		(A) edge  				node[below left] {1} 		(D)
		(B) edge  [loop above]	node {0} 		(B)
		(B) edge  				node {1} 		(C)
		(C) edge  [loop above]	node {0} 		(C)
		(C) edge  [bend left]	node {1} 		(A)
		(D) edge  				node {0} 		(E)
		(D) edge  [loop below]	node {1} 		(D)
		(E) edge  				node {1} 		(F)
		(F) edge  [loop below]	node {0,1} 		(F)
		;
  
\end{tikzpicture} 
\end{center}
\item Die Berechnungsbäume für die Wörter $w_1 = 0011$ und $w_2 = 11011$: 

\begin{multicols}{2}
\begin{center}
\begin{forest}
[{$(q_0,0011)$}
    [,phantom]
    [{$(q_0,011)$}
    	[,phantom]
    	[{$(q_0,11)$}
			[{$(q_3,1)$}
				[{$(q_3,\lambda)$}]
				[,phantom]
			]
			[{$(q_1,1)$}
				[,phantom]
				[{$(q_2,\lambda)$}]
			]
		]
    ]
]
\end{forest}
\end{center}
\columnbreak
\begin{center}
\begin{forest}
[{$(q_0,11001)$}
    [{$(q_3,1001)$}
    	[,phantom]
		[{$(q_3,001)$}
			[,phantom]
			[{$(q_4,01)$}]
		]
    ]
    [{$(q_1,1001)$}
    	[,phantom]
    	[{$(q_2,001)$}
			[,phantom]
			[{$(q_2,01)$}
				[,phantom]
				[{$(q_2,1)$}
					[,phantom]
					[{$(q_0,\lambda)$}]
				]
			]
		]
    ]
]
\end{forest}
\end{center}
\end{multicols}
\end{enumerate}
\end{homeworkProblem}

\clearpage

%----------------------------------------------------------------------------------------
%	PROBLEM 13
%----------------------------------------------------------------------------------------

\begin{homeworkProblem}

Wir konstruieren den deterministischen endlichen Automaten nach dem Beweis 3.2 im Buch. Die folgenden Schemata sind \emph{nicht} komplette endliche Automaten, sie zeigen lediglich den Arbeitsweg.

% Step 1 --------------------------------------------------------------
%\begin{center}
%\begin{tikzpicture}[->,>=stealth',shorten >=1pt,auto,node distance=2.8cm,
%                    semithick]
%  \tikzstyle{every state}=[fill=none,draw=black,text=black]
%  \node[initial,state]	(A)						{$\{p\}$};
%\end{tikzpicture} 
%\end{center}

% Step 2 --------------------------------------------------------------
\begin{center}
\begin{tikzpicture}[->,>=stealth',shorten >=1pt,auto,node distance=2.8cm,
                    semithick]
  \tikzstyle{every state}=[fill=none,draw=black,text=black]
  
  \node[initial,state]	(A)						{$\{p\}$};
  \node[state]			(B) [right of=A]		{$\{q\}$};
  
  \path (A) edge  [loop above]	node {a}		(A)
		(A) edge  				node {b} 		(B);
  
\end{tikzpicture} 
\end{center}

% Step 3 --------------------------------------------------------------
\begin{center}
\begin{tikzpicture}[->,>=stealth',shorten >=1pt,auto,node distance=2.8cm,
                    semithick]
  \tikzstyle{every state}=[fill=none,draw=black,text=black]
  
  \node[initial,state]	(A)						{$\{p\}$};
  \node[state]			(B) [right of=A]		{$\{q\}$};
  \node[state]			(C) [right of=B]		{$\{p,r\}$};
  
  \path (A) edge	[loop above]	node {a}		(A)
		(A) edge  					node {b} 		(B)
		(B) edge	[loop above]	node {b}		(B)
		(B) edge					node {a}		(C);
  
\end{tikzpicture} 
\end{center}

% Step 4 --------------------------------------------------------------
\begin{center}
\begin{tikzpicture}[->,>=stealth',shorten >=1pt,auto,node distance=2.8cm,
                    semithick]
  \tikzstyle{every state}=[fill=none,draw=black,text=black]
  
  \node[initial,state]	(A)						{$\{p\}$};
  \node[state]			(B) [right of=A]		{$\{q\}$};
  \node[state]			(C) [right of=B]		{$\{p,r\}$};
  \node[accepting,state](D) [right of=C]		{$\{p,r,s\}$};
  
  \path (A) edge	[loop above]	node {a}		(A)
		(A) edge  					node {b} 		(B)
		(B) edge	[loop above]	node {b}		(B)
		(B) edge	[bend left]		node {a}		(C)
		(C) edge					node {a}		(D)
		(C) edge	[bend left]		node {b}		(B)		
		;
  
\end{tikzpicture} 
\end{center}

% Step 5 --------------------------------------------------------------
\begin{center}
\begin{tikzpicture}[->,>=stealth',shorten >=1pt,auto,node distance=2.8cm,
                    semithick]
  \tikzstyle{every state}=[fill=none,draw=black,text=black]
  
  \node[initial,state]	(A)						{$\{p\}$};
  \node[state]			(B) [right of=A]		{$\{q\}$};
  \node[state]			(C) [right of=B]		{$\{p,r\}$};
  \node[accepting,state](D) [right of=C]		{$\{p,r,s\}$};
  \node[state]			(E) [right of=D]		{$\{r,q\}$};
  
  \path (A) edge	[loop above]	node {a}		(A)
		(A) edge  					node {b} 		(B)
		(B) edge	[loop above]	node {b}		(B)
		(B) edge	[bend left]		node {a}		(C)
		(C) edge					node {a}		(D)
		(C) edge	[bend left]		node {b}		(B)
		(D) edge	[loop above]	node {a}		(D)
		(D) edge					node {b}		(E)
		;
  
\end{tikzpicture} 
\end{center}


Der folgende deterministische endliche Automat ist äquivalent zum nichtdeterministischen endlichen Automat der Aufgabenstellung. 
% Step 6 --------------------------------------------------------------
\begin{center}
\begin{tikzpicture}[->,>=stealth',shorten >=1pt,auto,node distance=2.8cm,
                    semithick]
  \tikzstyle{every state}=[fill=none,draw=black,text=black]
  
  \node[initial,state]	(A)						{$\{p\}$};
  \node[state]			(B) [right of=A]		{$\{q\}$};
  \node[state]			(C) [right of=B]		{$\{p,r\}$};
  \node[accepting,state](D) [right of=C]		{$\{p,r,s\}$};
  \node[state]			(E) [right of=D]		{$\{r,q\}$};
  
  \path (A) edge	[loop above]	node {a}		(A)
		(A) edge  					node {b} 		(B)
		(B) edge	[loop above]	node {b}		(B)
		(B) edge	[bend left]		node {a}		(C)
		(C) edge					node {a}		(D)
		(C) edge	[bend left]		node[above] {b}		(B)
		(D) edge	[loop above]	node {a}		(D)
		(D) edge	[bend left]		node {b}		(E)
		(E) edge	[bend left]		node[above] {a}		(D)
		(E) edge	[bend left]		node {b}		(B)
		;
  
\end{tikzpicture} 
\end{center}

Die Funktionen $\delta$ und $\delta'$ sind folgendermassen definiert:
\begin{multicols}{2}
\begin{center}
\begin{tabular}{ c | c | c }
  $\delta$ & $a$ & $b$ \\
  \hline
  $p$ & $\{ p\}$ & $\{ q\}$ \\
  $q$ & $\{ p,r\}$ & $\{ q\}$ \\
  $r$ & $\{r,s\}$ & $\emptyset$ \\
  $s$ & $\{r,s\}$ & $\{ r\}$ \\
\end{tabular}
\end{center}

\columnbreak

\begin{center}
\begin{tabular}{ c | c | c }
  $\delta'$ & $a$ & $b$ \\
  \hline
  $\{p\}$ & $\{ p\}$ & $\{ q\}$ \\
  $\{q\}$ & $\{ p,r\}$ & $\{ q\}$ \\
  $\{p,r\}$ & $\{p,r,s\}$ & $\{ q\}$ \\
  $\{p,r,s\}$ & $\{p,r,s\}$ & $\{ r,a\}$ \\
  $\{r,q\}$ & $\{p,r,s\}$ & $\{ q\}$ \\
\end{tabular}
\end{center}
\end{multicols}

\end{homeworkProblem}

%----------------------------------------------------------------------------------------

\end{document}
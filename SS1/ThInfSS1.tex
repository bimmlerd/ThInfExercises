%%%%%%%%%%%%%%%%%%%%%%%%%%%%%%%%%%%%%%%%%
% Programming/Coding Assignment
% LaTeX Template
%
% Original author:
% Ted Pavlic (http://www.tedpavlic.com)
%
%
% This template uses a Perl script as an example snippet of code, most other
% languages are also usable. Configure them in the "CODE INCLUSION 
% CONFIGURATION" section.
%
%%%%%%%%%%%%%%%%%%%%%%%%%%%%%%%%%%%%%%%%%

%----------------------------------------------------------------------------------------
%	PACKAGES AND OTHER DOCUMENT CONFIGURATIONS
%----------------------------------------------------------------------------------------

\documentclass{article}
\usepackage[utf8]{inputenc}

\usepackage[swissgerman]{babel}
\usepackage{amsmath}
\usepackage{amsfonts}
\usepackage{fancyhdr} % Required for custom headers
\usepackage{lastpage} % Required to determine the last page for the footer
\usepackage{extramarks} % Required for headers and footers
\usepackage[usenames,dvipsnames]{color} % Required for custom colors
\usepackage{graphicx} % Required to insert images
\usepackage{listings} % Required for insertion of code
\usepackage{courier} % Required for the courier font
\usepackage{enumerate} % used for enumerate args
\usepackage{multicol} % columns

\usepackage{pgf} 
\usepackage{tikz}
\usepackage{forest} % treees :D
\usetikzlibrary{arrows,automata} %for FSM

% Custom commands
\DeclareMathOperator{\Kl}{Kl} %Klassen von Zuständen

\usepackage{mathtools}
\DeclarePairedDelimiter{\ceil}{\lceil}{\rceil}
% Shamelessly copied from http://tex.stackexchange.com/questions/43008/absolute-value-symbols
\DeclarePairedDelimiter\abs{\lvert}{\rvert} % nice |x|
\DeclarePairedDelimiter\norm{\lVert}{\rVert} % nice ||x||
% Swap the definition of \abs* and \norm*, so that \abs
% and \norm resizes the size of the brackets, and the 
% starred version does not.
\makeatletter
\let\oldabs\abs
\def\abs{\@ifstar{\oldabs}{\oldabs*}}
\let\oldnorm\norm
\def\norm{\@ifstar{\oldnorm}{\oldnorm*}}
\makeatother


% Margins
\topmargin=-0.45in
\evensidemargin=0in
\oddsidemargin=0in
\textwidth=6.5in
\textheight=9.0in
\headsep=0.25in

\linespread{1.1} % Line spacing

% Set up the header and footer
\pagestyle{fancy}
\lhead{\hmwkAuthorName} % Top left header
%\chead{\hmwkClass\ (\hmwkClassInstructor\): \hmwkTitle} % Top center head
%\rhead{\firstxmark} % Top right header
\rhead{}
\lfoot{\lastxmark} % Bottom left footer
\cfoot{} % Bottom center footer
\rfoot{Seite\ \thepage\ von\ \protect\pageref{LastPage}} % Bottom right footer
\renewcommand\headrulewidth{0.4pt} % Size of the header rule
\renewcommand\footrulewidth{0.4pt} % Size of the footer rule

\setlength\parindent{0pt} % Removes all indentation from paragraphs

%----------------------------------------------------------------------------------------
%	CODE INCLUSION CONFIGURATION
%----------------------------------------------------------------------------------------

\definecolor{MyDarkGreen}{rgb}{0.0,0.4,0.0} % This is the color used for comments
\lstloadlanguages{Pascal} % Load Pascal syntax for listings, for a list of other languages supported see: ftp://ftp.tex.ac.uk/tex-archive/macros/latex/contrib/listings/listings.pdf
\lstset{language=Perl, % Use Pascal in this example
        frame=single, % Single frame around code
        basicstyle=\small\ttfamily, % Use small true type font
        keywordstyle=[1]\color{Blue}\bf, % Pascal functions bold and blue
        keywordstyle=[2]\color{Purple}, % Pascal function arguments purple
        keywordstyle=[3]\color{Blue}\underbar, % Custom functions underlined and blue
        identifierstyle=, % Nothing special about identifiers                                         
        commentstyle=\usefont{T1}{pcr}{m}{sl}\color{MyDarkGreen}\small, % Comments small dark green courier font
        stringstyle=\color{Purple}, % Strings are purple
        showstringspaces=false, % Don't put marks in string spaces
        tabsize=5, % 5 spaces per tab
        %
        % Put standard Pascal functions not included in the default language here
        morekeywords={rand},
        %
        % Put Pascal function parameters here
        morekeywords=[2]{on, off, interp},
        %
        % Put user defined functions here
        morekeywords=[3]{test},
        %
        morecomment=[l][\color{Blue}]{...}, % Line continuation (...) like blue comment
        numbers=left, % Line numbers on left
        firstnumber=1, % Line numbers start with line 1
        numberstyle=\tiny\color{Blue}, % Line numbers are blue and small
        stepnumber=5 % Line numbers go in steps of 5
}

% Creates a new command to include a perl script, the first parameter is the filename of the script (without .p), the second parameter is the caption
\newcommand{\pascalscript}[2]{
\begin{itemize}
\item[]\lstinputlisting[caption=#2,label=#1]{#1.p}
\end{itemize}
}

%----------------------------------------------------------------------------------------
%	DOCUMENT STRUCTURE COMMANDS
%	Skip this unless you know what you're doing
%----------------------------------------------------------------------------------------

% Header and footer for when a page split occurs within a problem environment
%\newcommand{\enterProblemHeader}[1]{
%\nobreak\extramarks{#1}{#1 continued on next page\ldots}\nobreak
%\nobreak\extramarks{#1 (continued)}{#1 continued on next page\ldots}\nobreak
%}

% Header and footer for when a page split occurs between problem environments
%\newcommand{\exitProblemHeader}[1]{
%\nobreak\extramarks{#1 (continued)}{#1 continued on next page\ldots}\nobreak
%\nobreak\extramarks{#1}{}\nobreak
%}

\setcounter{secnumdepth}{0} % Removes default section numbers
\newcounter{homeworkProblemCounter} % Creates a counter to keep track of the number of problems

\newcommand{\homeworkProblemName}{}
\newenvironment{homeworkProblem}[1][Aufgabe \arabic{homeworkProblemCounter}]{ % Makes a new environment called homeworkProblem which takes 1 argument (custom name) but the default is "Problem #"
\stepcounter{homeworkProblemCounter} % Increase counter for number of problems
\renewcommand{\homeworkProblemName}{#1} % Assign \homeworkProblemName the name of the problem
\section{\homeworkProblemName} % Make a section in the document with the custom problem count
%\enterProblemHeader{\homeworkProblemName} % Header and footer within the environment
}{
%\exitProblemHeader{\homeworkProblemName} % Header and footer after the environment
}

\newcommand{\problemAnswer}[1]{ % Defines the problem answer command with the content as the only argument
\noindent\framebox[\columnwidth][c]{\begin{minipage}{0.98\columnwidth}#1\end{minipage}} % Makes the box around the problem answer and puts the content inside
}

\newcommand{\homeworkSectionName}{}
\newenvironment{homeworkSection}[1]{ % New environment for sections within homework problems, takes 1 argument - the name of the section
\renewcommand{\homeworkSectionName}{#1} % Assign \homeworkSectionName to the name of the section from the environment argument
\subsection{\homeworkSectionName} % Make a subsection with the custom name of the subsection
%\enterProblemHeader{\homeworkProblemName\ [\homeworkSectionName]} % Header and footer within the environment
}{
%\enterProblemHeader{\homeworkProblemName} % Header and footer after the environment
}

%----------------------------------------------------------------------------------------
%	NAME AND CLASS SECTION
%----------------------------------------------------------------------------------------

\newcommand{\hmwkTitle}{Selbststudium\ -\ Blatt\ 1} % Assignment title
\newcommand{\hmwkDueDate}{24.\ Oktober\ 2014} % Due date
\newcommand{\hmwkClass}{Theoretische Infomatik} % Course/class
\newcommand{\hmwkClassInstructor}{Prof. Hromkovič} % Teacher/lecturer
\newcommand{\hmwkAuthorName}{Vincent von Rotz, David Bimmler und Kevin Klein} % Your name

%----------------------------------------------------------------------------------------
%	TITLE PAGE
%----------------------------------------------------------------------------------------

\title{
\vspace{2in}
\textmd{\textbf{\hmwkClass:\ \hmwkTitle}}\\
\normalsize\vspace{0.1in}\small{Abgabe\ bis\ \hmwkDueDate}\\
\vspace{0.1in}\large{\textit{\hmwkClassInstructor}
\vspace{3in}
}}
\author{\textbf{\hmwkAuthorName}}
\date{} % Insert date here if you want it to appear below your name

%----------------------------------------------------------------------------------------

\begin{document}

\maketitle

%----------------------------------------------------------------------------------------
%	TABLE OF CONTENTS
%----------------------------------------------------------------------------------------

%\setcounter{tocdepth}{1} % Uncomment this line if you don't want subsections listed in the ToC

\addtocounter{homeworkProblemCounter}{0}
\newpage
%\tableofcontents
%\newpage

%----------------------------------------------------------------------------------------
%	PROBLEM 1
%----------------------------------------------------------------------------------------

\begin{homeworkProblem}
\begin{itemize}
\item \(G_1\) \\
The pure application of the derivation rules leads to the following:
\[XaYaZ \rightarrow a^{i}XaYaZ
	\rightarrow a^{i}baYaZ
	\rightarrow a^{i}bab^{j}YaZ
	\rightarrow a^{i}bab^{j}aaZ
	\rightarrow a^{i}bab^{j}aac^{k}Z
	\rightarrow a^{i}bab^{j}aac^{k}\]
Hence, let us define the language we would expect from this grammar:
\[ L_1 := \{ a^{i}bab^{j}aac^{k} \mid i,j,k \in \mathbb{N} \} \]
Let's prove \(L_1 = L(G_1)\) by proving mutual inclusions.

\begin{itemize}
	\item \(L_1 \subseteq L(G_1) \)   \\*
		\begin{gather*}
		w \in L_1 \Rightarrow \exists i,j,k \in \mathbb{N} \mid w =  a^{i}bab^{j}aac^{k} \\
		\text{Let's see if we can generate such a word in the scope of grammar \(G_1\).} \\
		S \rightarrow XaYaZ \text{ is applied anyway} \\
		\text{We can apply } X \rightarrow aX \text{ i times, followed by } X \rightarrow b \\
		XaYaZ \rightarrow a^{i}baYaZ \\
		\text{We can apply } Y \rightarrow bY \text{ j times, followed by } Y \rightarrow a \\
		a^{i}baYaZ \rightarrow a^{i}bab^{j}aaZ \\
		\text{We can apply } Z \rightarrow cZ \text{ k times, followed by } Z \rightarrow c \\
		a^{i}bab^{j}aaZ \rightarrow a^{i}bab^{j}aac^{k} \\
		\text{As we strictly applied the derivation rules of \(G_1\) for any word of \(L_1\), the inclusion is proven.}
		\end{gather*}
	\item \(L(G_1) \subseteq L_1 \)   \\*
		\begin{gather*}
		\text{Assume } w \in L(G_1) \\
		\text{We know that w has been generated by a finite number of applications of derivations rules, starting with S.} \\
		\text{There is only one possible derivation with S as left-hand side.} \\
		\rightarrow XaYaZ \\
		\text{X can only be replaced by either aX or b.} \\
		\text{It is obvious that any finite sequence of applications of those rules leads to something of the pattern \(a^{i}b\)} \\
		\text{Y can only be replaced by either bY or a.} \\
		\text{It is obvious that any finite sequence of applications of those rules leads to something of the pattern \(b^{j}a\)} \\
		\text{Z can only be replaced by either cZ or \(\lambda\) .} \\
		\text{It is obvious that any finite sequence of applications of those rules leads to something of the pattern \(c^{k}\)} \\
		\text{Putting it all together, this results in a word of the pattern \(a^{i}bab^{j}aac^{k}\), which is the exact definition of \(L_1\)}\\
		\end{gather*}
	\end{itemize}
\item \(G_2\)
	\begin{itemize}
 		\item
		\[\text{If }S \rightarrow 0 \text{ is applied, } w \in L(G_2) \Rightarrow w = 0\] \\
		\item
		\begin{gather*}
		\text{If not, } S \rightarrow X1X1X1XY \text{ is applied } \\
		X1X1X1XY \rightarrow 0^{i}10^{j}10^{k}10^{l}Y \\
		0^{i}10^{j}10^{k}10^{l}Y \rightarrow 0^{i}10^{j}10^{k}10^{l}[X1X1X1X1XY] \\
		0^{i}10^{j}10^{k}10^{l}[X1X1X1X1XY] \rightarrow 0^{i}10^{j}10^{k}10^{l}[0^{m}10^{n}10^{o}10^{p}10^{q}Y] \\
		\text{Y can be recursively replaced finitely many times.} \\
		A_1 := \{0\}^{\ast}, w_i \in A \\
		A_2 := \{w_{0}1w_{1}1w_{2}1w_{3}1w_{4}\}^{\ast}, A_3 \in A_2 \\
		\Rightarrow w \in A \cdot \{1\} \cdot A \cdot \{1\} \cdot A \cdot \{1\} \cdot A \cdot A_3  \\
		\Rightarrow L(G_2) =: L_2 = A \cdot \{1\} \cdot A \cdot \{1\} \cdot A \cdot \{1\} \cdot A \cdot A_3
		\end{gather*}
	\end{itemize}
\end{itemize}




%L_1
\end{homeworkProblem}
%----------------------------------------------------------------------------------------
%	PROBLEM 2
%----------------------------------------------------------------------------------------

\begin{homeworkProblem}
\begin{enumerate}[(a)]
\item
	Let's treat both conditions seperately. \\
	\begin{itemize}
	\item
		\( (|x|_a - |x|_b) \text{ mod } 3 = 1 \) \\
		Let X,Y,Z be non-terminals. \\
		\(X \rightarrow   (|x|_a - |x|_b) \text{ mod } 3 = 1 ;  Y \rightarrow   (|x|_a - |x|_b) \text{ mod } 3 = 2 ; Z \rightarrow   (|x|_a - |x|_b) \text{ mod } 3 = 0  \) \\
		Let's define the derivation rules.
		\[S \rightarrow aX ; S \rightarrow bY \] 
		\[X \rightarrow \lambda ; X \rightarrow aY ; X \rightarrow bZ\] 
		\[Y \rightarrow aZ ; Y \rightarrow bX \] 
		\[Z \rightarrow aX ; Z \rightarrow bY \] 
	\item
		\(bbb \in x\) \\
		\begin{gather*}
		S \rightarrow aA ; S \rightarrow bB_1 \\
		A \rightarrow aA ; A \rightarrow bB_1 \\
		B_1 \rightarrow aA ; B_1 \rightarrow bB_2 \\
		B_2 \rightarrow aA ; B_2 \rightarrow bB_3 \\
		B_3 \rightarrow aB_3 ; B_3 \rightarrow bB_3 ; B_3 \rightarrow \lambda \\
		\end{gather*}
	\end{itemize}
	P is the set containing all those rules.
\item
	Let's define the first rule. \\
	\(S \rightarrow 001A101D011 \) \\
	Let's treat both conditions seperately. \\
	\begin{itemize}
		\item \(|x| \text{ mod } 3 = 0 \) 
			\begin{gather*}
				A \rightarrow 0A ; A \rightarrow 1B ; A \rightarrow \lambda \\
				B \rightarrow 0B ; B \rightarrow 1C \\
				C \rightarrow 0C ; C \rightarrow 1A \\
			\end{gather*}
		\item \(|y|_0 \text{ mod } 2 = 1 \)
			\begin{gather*}
				D \rightarrow 1D ; D \rightarrow 0E \\
				E \rightarrow 1E ; E \rightarrow 0D ; E \rightarrow \lambda \\
			\end{gather*}
	\end{itemize}
	P is the set containing all those rules.
\end{enumerate}
\end{homeworkProblem}

%----------------------------------------------------------------------------------------
%	PROBLEM 3
%----------------------------------------------------------------------------------------

\begin{homeworkProblem}
In order to get to grammar regular and normed, the following derivation rules have to be adapted:
\begin{enumerate}[(1)]
\item \(S \rightarrow bS\)
\item \(S \rightarrow aaaX\)
\item \(Y \rightarrow bbZ\)
\item \(Y \rightarrow X\)
\item \(Y \rightarrow \lambda\)
%\item \(Z \rightarrow aZ\)
\end{enumerate}
They become
\begin{enumerate}[(1):]
\item \(S \rightarrow bH_0 ; H_0 \rightarrow bS ; H_0 \rightarrow aH_1 ; H_0 \rightarrow bY\)
\item \( S \rightarrow aH_1 ; H_1 \rightarrow aH_2 ; H_2 \rightarrow aX \)
\item \(Y \rightarrow bH_3 ; H_3 \rightarrow bZ\)
\item X only derivates to b, hence we can replace (4) by: \(Y \rightarrow b\)
\item There is only one rule with Y as right-hand side. Hence substituting Y by \(\lambda\) is that rule is sufficient: \(S \rightarrow b\)
\end{enumerate}
We also need $Z \to b$, since we had $Z\to bY\to\lambda$.
%(6): \(Z \rightarrow aH_4 ; H_4 \rightarrow aZ ; \) every other relation from H4 to something Z derivates to \\

\begin{center}
\begin{tikzpicture}[->,>=stealth',shorten >=1pt,auto,node distance=2.8cm,
                    semithick]
  \tikzstyle{every state}=[fill=none,draw=black,text=black]
  
  \node[initial,state] (A)                    {$S$};
  \node[state]         (B) [right of=A]		  {$H_1$};
  \node[state]         (C) [right of=B,node distance=1.5cm]		  {$H_2$};
  \node[state]         (D) [right of=C]		  {$X$};
  \node[accepting,state]         (E) [right of=D]			{$Q$};
  \node[accepting,state]         (F) [below right of=A]		{$Y$};
  \node[state]         (G) [below of=F,node distance=1.75cm]{$H_3$};
  \node[accepting,state]         (H) [below right of=F]		{$Z$};
  
  \path (A) edge  			 	node {a} (B)
  		(A) edge  [loop above] 	node {b} (A)
  		(A) edge  [bend left] 	node {b} (E)
		(A) edge				node {b} (F)
  		(B) edge				node {a} (C)
  		(C) edge				node {a} (D)
  		(D) edge				node {b} (E)
		(F) edge 				node {a} (D)
		(F) edge 				node[below] {b} (E)
		(F) edge 				node {b} (G)
		(G) edge				node[above] {b} (H)
		(H) edge 				node[above right] {b} (F)
		(H) edge 				node[below] {a,b} (E)
		(H) edge [loop below]	node {a} (H);
\end{tikzpicture} 
\end{center}

\end{homeworkProblem}



%----------------------------------------------------------------------------------------

\end{document}
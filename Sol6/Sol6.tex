%%%%%%%%%%%%%%%%%%%%%%%%%%%%%%%%%%%%%%%%%
% Programming/Coding Assignment
% LaTeX Template
%
% Original author:
% Ted Pavlic (http://www.tedpavlic.com)
%
%
% This template uses a Perl script as an example snippet of code, most other
% languages are also usable. Configure them in the "CODE INCLUSION 
% CONFIGURATION" section.
%
%%%%%%%%%%%%%%%%%%%%%%%%%%%%%%%%%%%%%%%%%

%----------------------------------------------------------------------------------------
%	PACKAGES AND OTHER DOCUMENT CONFIGURATIONS
%----------------------------------------------------------------------------------------

\documentclass{article}
\usepackage[utf8]{inputenc}

\usepackage[swissgerman]{babel}
\usepackage{amsmath}
\usepackage{amssymb}
\usepackage{amsfonts}
\usepackage{fancyhdr} % Required for custom headers
\usepackage{lastpage} % Required to determine the last page for the footer
\usepackage{extramarks} % Required for headers and footers
\usepackage[usenames,dvipsnames]{color} % Required for custom colors
\usepackage{graphicx} % Required to insert images
\usepackage{listings} % Required for insertion of code
\usepackage{courier} % Required for the courier font
\usepackage{enumerate} % used for enumerate args
\usepackage{multicol} % columns

\usepackage{pgf} 
\usepackage{tikz}
\usepackage{forest} % treees :D
\usetikzlibrary{arrows,automata} %for FSM

% Custom commands
\DeclareMathOperator{\Kl}{Kl} %Klassen von Zuständen
\newcommand{\N}{\mathbb{N}}
\newcommand{\R}{\mathbb{R}}
\newcommand{\C}{\mathbb{C}}
\newcommand{\Oh}{\mathcal{O}}
\newcommand{\Lcl}{\mathcal{L}}
\newcommand{\Th}{\Theta}

\usepackage{mathtools}
\DeclarePairedDelimiter{\ceil}{\lceil}{\rceil}
% Shamelessly copied from http://tex.stackexchange.com/questions/43008/absolute-value-symbols
\DeclarePairedDelimiter\abs{\lvert}{\rvert} % nice |x|
\DeclarePairedDelimiter\norm{\lVert}{\rVert} % nice ||x||
% Swap the definition of \abs* and \norm*, so that \abs
% and \norm resizes the size of the brackets, and the 
% starred version does not.
\makeatletter
\let\oldabs\abs
\def\abs{\@ifstar{\oldabs}{\oldabs*}}
\let\oldnorm\norm
\def\norm{\@ifstar{\oldnorm}{\oldnorm*}}
\makeatother


% Margins
\topmargin=-0.45in
\evensidemargin=0in
\oddsidemargin=0in
\textwidth=6.5in
\textheight=9.0in
\headsep=0.25in

\linespread{1.1} % Line spacing

% Set up the header and footer
\pagestyle{fancy}
\lhead{\hmwkAuthorName} % Top left header
%\chead{\hmwkClass\ (\hmwkClassInstructor\): \hmwkTitle} % Top center head
%\rhead{\firstxmark} % Top right header
\rhead{}
\lfoot{\lastxmark} % Bottom left footer
\cfoot{} % Bottom center footer
\rfoot{Seite\ \thepage\ von\ \protect\pageref{LastPage}} % Bottom right footer
\renewcommand\headrulewidth{0.4pt} % Size of the header rule
\renewcommand\footrulewidth{0.4pt} % Size of the footer rule

\setlength\parindent{0pt} % Removes all indentation from paragraphs

%----------------------------------------------------------------------------------------
%	CODE INCLUSION CONFIGURATION
%----------------------------------------------------------------------------------------

\definecolor{MyDarkGreen}{rgb}{0.0,0.4,0.0} % This is the color used for comments
\lstloadlanguages{Pascal} % Load Pascal syntax for listings, for a list of other languages supported see: ftp://ftp.tex.ac.uk/tex-archive/macros/latex/contrib/listings/listings.pdf
\lstset{language=Perl, % Use Pascal in this example
        frame=single, % Single frame around code
        basicstyle=\small\ttfamily, % Use small true type font
        keywordstyle=[1]\color{Blue}\bf, % Pascal functions bold and blue
        keywordstyle=[2]\color{Purple}, % Pascal function arguments purple
        keywordstyle=[3]\color{Blue}\underbar, % Custom functions underlined and blue
        identifierstyle=, % Nothing special about identifiers                                         
        commentstyle=\usefont{T1}{pcr}{m}{sl}\color{MyDarkGreen}\small, % Comments small dark green courier font
        stringstyle=\color{Purple}, % Strings are purple
        showstringspaces=false, % Don't put marks in string spaces
        tabsize=5, % 5 spaces per tab
        %
        % Put standard Pascal functions not included in the default language here
        morekeywords={rand},
        %
        % Put Pascal function parameters here
        morekeywords=[2]{on, off, interp},
        %
        % Put user defined functions here
        morekeywords=[3]{test},
        %
        morecomment=[l][\color{Blue}]{...}, % Line continuation (...) like blue comment
        numbers=left, % Line numbers on left
        firstnumber=1, % Line numbers start with line 1
        numberstyle=\tiny\color{Blue}, % Line numbers are blue and small
        stepnumber=5 % Line numbers go in steps of 5
}

% Creates a new command to include a perl script, the first parameter is the filename of the script (without .p), the second parameter is the caption
\newcommand{\pascalscript}[2]{
\begin{itemize}
\item[]\lstinputlisting[caption=#2,label=#1]{#1.p}
\end{itemize}
}

%----------------------------------------------------------------------------------------
%	DOCUMENT STRUCTURE COMMANDS
%	Skip this unless you know what you're doing
%----------------------------------------------------------------------------------------

% Header and footer for when a page split occurs within a problem environment
%\newcommand{\enterProblemHeader}[1]{
%\nobreak\extramarks{#1}{#1 continued on next page\ldots}\nobreak
%\nobreak\extramarks{#1 (continued)}{#1 continued on next page\ldots}\nobreak
%}

% Header and footer for when a page split occurs between problem environments
%\newcommand{\exitProblemHeader}[1]{
%\nobreak\extramarks{#1 (continued)}{#1 continued on next page\ldots}\nobreak
%\nobreak\extramarks{#1}{}\nobreak
%}

\setcounter{secnumdepth}{0} % Removes default section numbers
\newcounter{homeworkProblemCounter} % Creates a counter to keep track of the number of problems

\newcommand{\homeworkProblemName}{}
\newenvironment{homeworkProblem}[1][Aufgabe \arabic{homeworkProblemCounter}]{ % Makes a new environment called homeworkProblem which takes 1 argument (custom name) but the default is "Problem #"
\stepcounter{homeworkProblemCounter} % Increase counter for number of problems
\renewcommand{\homeworkProblemName}{#1} % Assign \homeworkProblemName the name of the problem
\section{\homeworkProblemName} % Make a section in the document with the custom problem count
%\enterProblemHeader{\homeworkProblemName} % Header and footer within the environment
}{
%\exitProblemHeader{\homeworkProblemName} % Header and footer after the environment
}

\newcommand{\problemAnswer}[1]{ % Defines the problem answer command with the content as the only argument
\noindent\framebox[\columnwidth][c]{\begin{minipage}{0.98\columnwidth}#1\end{minipage}} % Makes the box around the problem answer and puts the content inside
}

\newcommand{\homeworkSectionName}{}
\newenvironment{homeworkSection}[1]{ % New environment for sections within homework problems, takes 1 argument - the name of the section
\renewcommand{\homeworkSectionName}{#1} % Assign \homeworkSectionName to the name of the section from the environment argument
\subsection{\homeworkSectionName} % Make a subsection with the custom name of the subsection
%\enterProblemHeader{\homeworkProblemName\ [\homeworkSectionName]} % Header and footer within the environment
}{
%\enterProblemHeader{\homeworkProblemName} % Header and footer after the environment
}

%----------------------------------------------------------------------------------------
%	NAME AND CLASS SECTION
%----------------------------------------------------------------------------------------

\newcommand{\hmwkTitle}{Übungsaufgaben\ 6} % Assignment title
\newcommand{\hmwkDueDate}{07.\ November\ 2014} % Due date
\newcommand{\hmwkClass}{Theoretische Informatik} % Course/class
\newcommand{\hmwkClassInstructor}{Prof. Welzl} % Teacher/lecturer
\newcommand{\hmwkAuthorName}{Vincent von Rotz, David Bimmler und Kevin Klein} % Your name

%----------------------------------------------------------------------------------------
%	TITLE PAGE
%----------------------------------------------------------------------------------------

\title{
\vspace{2in}
\textmd{\textbf{\hmwkClass:\ \hmwkTitle}}\\
\normalsize\vspace{0.1in}\small{Abgabe\ bis\ \hmwkDueDate}\\
\vspace{0.1in}\large{\textit{\hmwkClassInstructor}
\vspace{3in}
}}
\author{\textbf{\hmwkAuthorName}}
\date{} % Insert date here if you want it to appear below your name

%----------------------------------------------------------------------------------------

\begin{document}

\maketitle

%----------------------------------------------------------------------------------------
%	TABLE OF CONTENTS
%----------------------------------------------------------------------------------------

%\setcounter{tocdepth}{1} % Uncomment this line if you don't want subsections listed in the ToC

\addtocounter{homeworkProblemCounter}{16}
\newpage
%\tableofcontents
%\newpage

%----------------------------------------------------------------------------------------
%	Aufgabe 17
%----------------------------------------------------------------------------------------

\begin{homeworkProblem}
	\begin{enumerate}[(a)]
	\item \((L_{infinite})^\complement \in \mathcal{L}_{RE}\) \\
		Offensichtlich gilt: \\
	\((L_{infinite})^\complement = \{ x \mid \nexists TM\ M s.t.\ x=Kod(M)\#w \vee \exists TM\ M\  s.t. x=Kod(M)\#w \wedge M\) hält auf 	mindestens einer Eingabe\} \\
		Wir entewerfen eine NTM M, die das Entscheidungsproblem \((( L_{infinite})^\complement , \Sigma)\) löst. \\
		Dies impliziert  \((L_{infinite})^\complement \in \mathcal{L}_{RE}\), da es laut Satz 4.2 für jede NTM eine aequivalente TM gibt mit. \\
		Die NTM sollte folgend auf einem Input \(x\) funktionnieren: 
		\begin{enumerate}[-]
		\item Wenn w nicht die Kodierung einer TM ist, wird \(x\) akzeptiert.
		\item Wenn w die Kodierung einer TM ist:
			\begin{enumerate}[1.]
			\item Suche nichtdeterministisch ein \(y\).
			\item Entscheide deterministisch ob \(x=Kod(M) y\) akzeptiert indem \(M\) auf \(y\) simuliert wird.
				\begin{enumerate}[-]
				\item Wenn \(y\) von \(M\) akzeptiert wird, wird x akzeptiert. 
				\item Wenn \(y\) vom \(M\) nicht akzeptiert wird, wird dieses y verworfen.
				\end{enumerate}
			\end{enumerate}
		\end{enumerate}
	\item \((L_{infinite})^\complement \notin \mathcal{L}_{R}\) \\
		Wenn wir zeigen können, dass \(L_H  \leq_{RR} (L_{infinite})^\complement\) folgt, dass \((L_{infiinte})^\complement \notin \mathcal{L}_R\). \\
		Zur EE-Reduzierbarkeit verwenden wir eine TM \(M'\), \((L_{infinite})^\complement\) auf \(L_H\) reduziert und \(x\) zu \(f_{M'}(x)=Kod(B_x)\) transformiert. \\
		\(M'\) generiert die Kodierung der TM \(B_x\), welche wie folgt definiert ist: \\
		Falls:
		\begin{enumerate}[-]
		\item \(\exists TM\ M\ s.t.\ x=Kod(M)\#w\) \\
			\(B_x\) generiert \(Kod(M)\) und \(w\) s.t. \(x=Kod(M)\#w\). \\
			\(B_x\) simuliert M auf w. \(B_x\) übernimmt die Ausgabe von M, dies ist jedoch irrelevant.\\
		\item \(\nexists TM\ M\ s.t.\ x=Kod(M)\#w\) \\
			\(B_x\) ist eine beliebige TM die auf keiner Eingabe hält.
		\end{enumerate}
		Es gilt offenbar: \\
		\begin{equation} \begin{split}
		x \in L_H &\Longleftrightarrow x = Mod(M)\#w \wedge M \text{ hält auf } w\\
		&\Longleftrightarrow B_x \text{ generiert } M \text{ auf } w \text{ und hält, da } M \text{ auf } w \text{ hält.} \\
		&\Longleftrightarrow Kod(B_x) \in (L_{infinite})^\complement  \\
		\end{split} \end{equation}
	\item \(L_{infinite} \notin \mathcal{L}_{RE}\) \\
	Lemma 5.4 besagt, dass \(L \leq_{R} L^\complement\) und \(L^\complement \leq_{R} L\).\\
	Daraus folgt mit (b) direkt, dass \(L_{infinite} \notin \Lcl_R\). \\
	\end{enumerate}
\end{homeworkProblem}
%----------------------------------------------------------------------------------------
%	Aufgabe 18
%----------------------------------------------------------------------------------------

\begin{homeworkProblem}
	\begin{enumerate}[(a)]
	\item
	\item \(L_{all} \notin \mathcal(L)_{empty} \) \\
		Wir zeigen dies in dem wir die EE-Reduktion von \(L_{empty}\) auf \(L_{all}\) mittels einer TM M zeigen.\\
		Falls: \\
		\begin{enumerate}[-]
		\item \(\nexists TM\ M\ s.t.\ x=Kod(M)\) \\
			Übergebe \(x\) an \(L_{empty} (f_M(x) = x)\)\\
		\item \(\exists TM\ M\ s.t.\ x=Kod(M)\) \\
			Die TM \(M'\) generiert die Kodierung einer TM B. \\
			B ist die Sprache, die alle Wörter akzeptiert, die von \(M\) nicht akzeptiert werden. \\ 
		\end{enumerate}
		Es gilt offenbar: \\
		\begin{equation} \begin{split}
		x \in L_{all} &\Longleftrightarrow \exists TM\ M\ s.t. x=Kod(M) \wedge L(M) = (\Sigma_{bool})^\ast \\
		&\Longleftrightarrow L(B) = \emptyset \\
		&\Longleftrightarrow Kod(B) \in L_{empty} \\
		\end{split} \end{equation}	
	\end{enumerate}
\end{homeworkProblem}

%----------------------------------------------------------------------------------------
%	Aufgabe 19
%----------------------------------------------------------------------------------------

\begin{homeworkProblem}
Wir betrachten eine unendliche Sprache $L$. Da wir jedes Wort über ein Alphabet $\Sigma$ auf $\Sigma_{bool}$ abbilden können, sei ohne Beschränkung der Allgemeinheit sei das Alphabet der Sprache $L$ $\Sigma_{bool}$. Wir kennen die kanonische Ordnung über $\Sigma_{bool}$. Wir können also die Wörter aus $L$ kanonisch geordnet aufzählen. Sei also $w_i$ das kanonisch $i$-te Wort aus $L$. Ausserdem können wir die Turingmaschinen aufzählen. Wir erstellen jetzt eine unendliche Tabelle mit $w_1, w_2, \dots$ als Spalten, und den Turingmaschinen $M_1, M_2, \dots$ als Zeilen.
\renewcommand{\arraystretch}{1.2}
\begin{center}
    \begin{tabular}{  c | c  c  c  c  c  c }
    \hline
     & $w_1$ & $w_2$ & $w_3$ & $\dots$ & $w_i$ & $\dots$ \\ \hline
    $M_1$ & \fbox{$d_{11}$} & $d_{12}$ & $d_{13}$ & $\dots$ & $d_{1i}$ & $\dots$ \\ 
    $M_2$ & $d_{21}$ & \fbox{$d_{22}$} & $d_{23}$ & $\dots$ & $d_{2i}$ & $\dots$ \\ 	
    $M_3$ & $d_{31}$ & $d_{32}$ & \fbox{$d_{33}$} & $\dots$ & $d_{3i}$ & $\dots$ \\ 
    $\vdots$ & $\vdots$ & $\vdots$ & $\vdots$ &  & $\vdots$ &  \\ 
    $M_i$ & $d_{i1}$ & $d_{i2}$ & $d_{i3}$ & $\dots$ & \fbox{$d_{ii}$} & $\dots$ \\ 
    $\vdots$ & $\vdots$ & $\vdots$ & $\vdots$ &  & $\vdots$ &  \\ 
    \hline
    \end{tabular}
\end{center}
\renewcommand{\arraystretch}{1.0}

Wir konstruieren jetzt eine Sprache $L_{diag}'$, die keiner der Sprachen $L(M_i)$ entspricht und eine Teilmenge von $L$ ist. Wir definieren
\[ L_{diag}' = \{ w \in (\Sigma_{bool})^* \mid w = w_i \text{ für ein } i \in \N - \{0\} \text{ und } d_{ii} = 0 \} \]

Wir beweisen $L_{diag}'\not\in \Lcl_{RE}$ indirekt. Sei $L_{diag}'\in \Lcl_{RE}$. Dann ist $L_{diag}' = L(M)$ für eine TM $M$. Weil $M$ eine der Turingmaschinen in der Nummerierung aller Turingmaschinen sein muss, existiert ein $i \in \N - \{ 0\}$, so dass $M=M_i$. Aber $L_{diag}'$ kann nicht gleich $L(M_i)$ sein, weil folgende Äquivalenz gilt:
\[ w_i \in L_{diag}' \Longleftrightarrow d_{ii} = 0 \Longleftrightarrow  w_i \not\in L_{diag}' \]
d.h., $w_i$ ist genau in einer der Sprachen $L_{diag}'$ oder $L(M_i)$. Damit gilt $L_{diag}' \not\in \Lcl_{RE}$ und wir haben eine Teilmenge der unendlichen Sprache $L$ gefunden, die nicht in $\Lcl_{RE}$ ist.
\end{homeworkProblem}



%----------------------------------------------------------------------------------------

\end{document}
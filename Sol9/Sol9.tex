%%%%%%%%%%%%%%%%%%%%%%%%%%%%%%%%%%%%%%%%%
% Programming/Coding Assignment
% LaTeX Template
%
% Original author:
% Ted Pavlic (http://www.tedpavlic.com)
%
%
% This template uses a Perl script as an example snippet of code, most other
% languages are also usable. Configure them in the "CODE INCLUSION 
% CONFIGURATION" section.
%
%%%%%%%%%%%%%%%%%%%%%%%%%%%%%%%%%%%%%%%%%

%----------------------------------------------------------------------------------------
%	PACKAGES AND OTHER DOCUMENT CONFIGURATIONS
%----------------------------------------------------------------------------------------

\documentclass{article}
\usepackage[utf8]{inputenc}

\usepackage[swissgerman]{babel}
\usepackage{amsmath}
\usepackage{amsfonts}
\usepackage{fancyhdr} % Required for custom headers
\usepackage{lastpage} % Required to determine the last page for the footer
\usepackage{extramarks} % Required for headers and footers
\usepackage[usenames,dvipsnames]{color} % Required for custom colors
\usepackage{graphicx} % Required to insert images
\usepackage{listings} % Required for insertion of code
\usepackage{courier} % Required for the courier font
\usepackage{enumerate} % used for enumerate args
\usepackage{multicol} % columns

\usepackage{pgf} 
\usepackage{tikz}
\usepackage{forest} % treees :D
\usetikzlibrary{arrows,automata} %for FSM

% Custom commands
\DeclareMathOperator{\Kl}{Kl} %Klassen von Zuständen
\newcommand{\N}{\mathbb{N}}
\newcommand{\R}{\mathbb{R}}
\newcommand{\C}{\mathbb{C}}
\newcommand{\Oh}{\mathcal{O}}
\newcommand{\Lcl}{\mathcal{L}}
\newcommand{\Th}{\Theta}
\newcommand{\NTIME}{\text{NTIME}}
\newcommand{\TIME}{\text{TIME}}
\newcommand{\Time}{\text{Time}}
\newcommand{\NSPACE}{\text{NSPACE}}
\newcommand{\SPACE}{\text{SPACE}}
\newcommand{\Space}{\text{Space}}

\usepackage{mathtools}
\DeclarePairedDelimiter{\ceil}{\lceil}{\rceil}
% Shamelessly copied from http://tex.stackexchange.com/questions/43008/absolute-value-symbols
\DeclarePairedDelimiter\abs{\lvert}{\rvert} % nice |x|
\DeclarePairedDelimiter\norm{\lVert}{\rVert} % nice ||x||
% Swap the definition of \abs* and \norm*, so that \abs
% and \norm resizes the size of the brackets, and the 
% starred version does not.
\makeatletter
\let\oldabs\abs
\def\abs{\@ifstar{\oldabs}{\oldabs*}}
\let\oldnorm\norm
\def\norm{\@ifstar{\oldnorm}{\oldnorm*}}
\makeatother


% Margins
\topmargin=-0.45in
\evensidemargin=0in
\oddsidemargin=0in
\textwidth=6.5in
\textheight=9.0in
\headsep=0.25in

\linespread{1.1} % Line spacing

% Set up the header and footer
\pagestyle{fancy}
\lhead{\hmwkAuthorName} % Top left header
%\chead{\hmwkClass\ (\hmwkClassInstructor\): \hmwkTitle} % Top center head
%\rhead{\firstxmark} % Top right header
\rhead{}
\lfoot{\lastxmark} % Bottom left footer
\cfoot{} % Bottom center footer
\rfoot{Seite\ \thepage\ von\ \protect\pageref{LastPage}} % Bottom right footer
\renewcommand\headrulewidth{0.4pt} % Size of the header rule
\renewcommand\footrulewidth{0.4pt} % Size of the footer rule

\setlength\parindent{0pt} % Removes all indentation from paragraphs


%----------------------------------------------------------------------------------------
%	CODE INCLUSION CONFIGURATION
%----------------------------------------------------------------------------------------

\definecolor{MyDarkGreen}{rgb}{0.0,0.4,0.0} % This is the color used for comments
\lstloadlanguages{Pascal} % Load Pascal syntax for listings, for a list of other languages supported see: ftp://ftp.tex.ac.uk/tex-archive/macros/latex/contrib/listings/listings.pdf
\lstset{language=Perl, % Use Pascal in this example
        frame=single, % Single frame around code
        basicstyle=\small\ttfamily, % Use small true type font
        keywordstyle=[1]\color{Blue}\bf, % Pascal functions bold and blue
        keywordstyle=[2]\color{Purple}, % Pascal function arguments purple
        keywordstyle=[3]\color{Blue}\underbar, % Custom functions underlined and blue
        identifierstyle=, % Nothing special about identifiers                                         
        commentstyle=\usefont{T1}{pcr}{m}{sl}\color{MyDarkGreen}\small, % Comments small dark green courier font
        stringstyle=\color{Purple}, % Strings are purple
        showstringspaces=false, % Don't put marks in string spaces
        tabsize=5, % 5 spaces per tab
        %
        % Put standard Pascal functions not included in the default language here
        morekeywords={rand},
        %
        % Put Pascal function parameters here
        morekeywords=[2]{on, off, interp},
        %
        % Put user defined functions here
        morekeywords=[3]{test},
        %
        morecomment=[l][\color{Blue}]{...}, % Line continuation (...) like blue comment
        numbers=left, % Line numbers on left
        firstnumber=1, % Line numbers start with line 1
        numberstyle=\tiny\color{Blue}, % Line numbers are blue and small
        stepnumber=5 % Line numbers go in steps of 5
}

% Creates a new command to include a perl script, the first parameter is the filename of the script (without .p), the second parameter is the caption
\newcommand{\pascalscript}[2]{
\begin{itemize}
\item[]\lstinputlisting[caption=#2,label=#1]{#1.p}
\end{itemize}
}

%----------------------------------------------------------------------------------------
%	DOCUMENT STRUCTURE COMMANDS
%	Skip this unless you know what you're doing
%----------------------------------------------------------------------------------------

% Header and footer for when a page split occurs within a problem environment
%\newcommand{\enterProblemHeader}[1]{
%\nobreak\extramarks{#1}{#1 continued on next page\ldots}\nobreak
%\nobreak\extramarks{#1 (continued)}{#1 continued on next page\ldots}\nobreak
%}

% Header and footer for when a page split occurs between problem environments
%\newcommand{\exitProblemHeader}[1]{
%\nobreak\extramarks{#1 (continued)}{#1 continued on next page\ldots}\nobreak
%\nobreak\extramarks{#1}{}\nobreak
%}

\setcounter{secnumdepth}{0} % Removes default section numbers
\newcounter{homeworkProblemCounter} % Creates a counter to keep track of the number of problems

\newcommand{\homeworkProblemName}{}
\newenvironment{homeworkProblem}[1][Aufgabe \arabic{homeworkProblemCounter}]{ % Makes a new environment called homeworkProblem which takes 1 argument (custom name) but the default is "Problem #"
\stepcounter{homeworkProblemCounter} % Increase counter for number of problems
\renewcommand{\homeworkProblemName}{#1} % Assign \homeworkProblemName the name of the problem
\section{\homeworkProblemName} % Make a section in the document with the custom problem count
%\enterProblemHeader{\homeworkProblemName} % Header and footer within the environment
}{
%\exitProblemHeader{\homeworkProblemName} % Header and footer after the environment
}

\newcommand{\problemAnswer}[1]{ % Defines the problem answer command with the content as the only argument
\noindent\framebox[\columnwidth][c]{\begin{minipage}{0.98\columnwidth}#1\end{minipage}} % Makes the box around the problem answer and puts the content inside
}

\newcommand{\homeworkSectionName}{}
\newenvironment{homeworkSection}[1]{ % New environment for sections within homework problems, takes 1 argument - the name of the section
\renewcommand{\homeworkSectionName}{#1} % Assign \homeworkSectionName to the name of the section from the environment argument
\subsection{\homeworkSectionName} % Make a subsection with the custom name of the subsection
%\enterProblemHeader{\homeworkProblemName\ [\homeworkSectionName]} % Header and footer within the environment
}{
%\enterProblemHeader{\homeworkProblemName} % Header and footer after the environment
}

%----------------------------------------------------------------------------------------
%	NAME AND CLASS SECTION
%----------------------------------------------------------------------------------------

\newcommand{\hmwkTitle}{Übungsaufgaben\ -\ Blatt\ 9} % Assignment title
\newcommand{\hmwkDueDate}{05.\ Dezember\ 2014} % Due date
\newcommand{\hmwkClass}{Theoretische Informatik} % Course/class
\newcommand{\hmwkClassInstructor}{Prof. Welzl} % Teacher/lecturer
\newcommand{\hmwkAuthorName}{David Bimmler, Vincent von Rotz und Kevin Klein} % Your name

%----------------------------------------------------------------------------------------
%	TITLE PAGE
%----------------------------------------------------------------------------------------

\title{
\vspace{2in}
\textmd{\textbf{\hmwkClass:\ \hmwkTitle}}\\
\normalsize\vspace{0.1in}\small{Abgabe\ bis\ \hmwkDueDate}\\
\vspace{0.1in}\large{\textit{\hmwkClassInstructor}
\vspace{3in}
}}
\author{\textbf{\hmwkAuthorName}}
\date{} % Insert date here if you want it to appear below your name

%----------------------------------------------------------------------------------------

\begin{document}

\maketitle

%----------------------------------------------------------------------------------------
%	TABLE OF CONTENTS
%----------------------------------------------------------------------------------------

%\setcounter{tocdepth}{1} % Uncomment this line if you don't want subsections listed in the ToC

\addtocounter{homeworkProblemCounter}{26}
\newpage
%\tableofcontents
%\newpage

%----------------------------------------------------------------------------------------
%	PROBLEM 27
%----------------------------------------------------------------------------------------

\begin{homeworkProblem}
Überlegen wir uns zunächst, wie wir anhand der gegebenen Elemente aus \( S = \{ r_i, {r_i}', s_i, {s_i}' \} \) zu $t$ summieren können. \\
$t$ ist so definiert, dass
\[ \forall l \in [1,n],\ t[l] = 1 \]
Da für $i=l$ $r_i$ und ${r_i}'$ so definiert sind, dass beide an Stelle $l$ 1 sind, folgt unmittelbar
\[ \forall i \in [1,n],\ r_i \in U \oplus {r_i}' \in U  \text{ (1)}\]
Die Definition von $t$ besagt ebenfalls, dass
\[\forall l \in [n+1, n+k],\ t[l]=4 \]
Wir bemerken, dass wir für jedes $l$, eine Summe in $[1,3]$ erreichen können, indem wir $s_l$ und ${s_l}'$ in $U$ einfügen. Um eine Summe von 4 zu erreichen, müssen wir also für jedes l eine Summe in $[1,3]$ durch das Hinzunehmen von $r_i$ bzw. ${r_i}'$ erreichen. Mit anderen Worten müssen wir also für jede Stelle l ein bis drei $r_i$ bzw. ${r_i}'$ mit $r_i[l] = 1$ bzw. ${r_i}'[l] = 1$ in $U$ haben. (2)\\

%anderer Ansatz
%\( \Phi = C_1 \wedge C_2 \wedge ... \wedge C_k \text{ ist erfüllbar } \Leftrightarrow \forall l \in [1,k]\  \exists x_i %\text{ mit } x_i = 1 \text{ wobei nicht } x_i = 1\ \wedge\ x_i = 0 \text{ gleichzeitig gelten kann} \)
%anderer Ansatz Ende

Dank (1) und (2) folgt: \\
\( \sum_{x \in U} x = t \Leftrightarrow ( \forall i \in [1..n],\ r_i[i] = 1 \oplus {r_i}'[i] = 1 ) \wedge (\forall l \in [n+1,n+k],\ \exists\ r_i\ \text{(bzw. ${r_i}'$ ) } s.t.\ r_i[l] = 1) \) \\
\( \Leftrightarrow  (\forall i \in [1,n]\ r_i \in U \oplus {r_i}' \in U) \wedge (\forall l \in [n+1,n+k],\ \exists\ x_i\ \text{(bzw. $\overline{x_i}$ ) } s.t.\ x_i \text{ kommt in Klausel $C_{l-n}$ vor}) \) \\
\( \Leftrightarrow  (\forall i \in [1,n]\ r_i \in U \oplus {r_i}' \in U) \wedge (\forall l \in [1,k],\ \exists\ x_i\ \text{(bzw. $\overline{x_i}$ ) } s.t.\ x_i \text{ kommt in Klausel $C_{l}$ vor}) \) \\
\( \Leftrightarrow  (\forall x_i,\ r_i \in U \oplus {r_i}' \in U) \wedge (\forall l \in [1,k],\ \exists\ x_i\ \text{(bzw. $\overline{x_i}$ ) } s.t.\ x_i \text{ kommt in Klausel $C_{l}$ vor}) \)
Wir nehmen immer entweder $r_i$ oder ${r_i}'$, also können wir $r_i$ als Wahrheitswert für $x_i$ betrachten. Falls $x_i = 1$ nehmen wir $r_i \in U$, falls $x_i = 0$ nemen wir ${r_i}' \in U$. \\
\( \Leftrightarrow (\forall x_i, x_i = 1 \oplus x_i = 0 ) \wedge (\forall l \in [1,k],\ \exists\ x_i\ \text{(bzw. $\overline{x_i}$ ) } s.t.\ x_i \text{ kommt in Klausel $C_{l}$ vor}) \) \\
Damit eine Formel erfüllbar ist, ist sie nicht widersprüchlich ( \(  !\exists x_i\ s.t.\ x_i = 1, \overline{x_i} = 1 \) ) und muss man in jeder Klausel eine Variabel für die Erfüllung der Klausel verantwortlich machen können.  \\
% Nicht sicher ob klar dass \exists x_i in C_l \implies man kann \x_i verantworrlich machen, glaub aber schon weil man die i's ja theoretisch beliebig setzen kann
\( \Leftrightarrow (\forall x_i, x_i = 1 \oplus x_i = 0 ) \wedge (\forall l \in [1,k],\ \exists\ x_i\ \text{(bzw. $\overline{x_i}$ ) } s.t.\ x_i \text{ erfüllt Klausel $C_{l}$ }) \) \\
\( \Leftrightarrow  \Phi = C_1 \wedge C_2 \wedge ... \wedge C_k \text{ ist erfüllbar } \)
\end{homeworkProblem}
%----------------------------------------------------------------------------------------
%	PROBLEM 28
%----------------------------------------------------------------------------------------

\begin{homeworkProblem}
Wir zeigen, dass PARTITION NP-vollständig ist, indem wir die Reduktion SUBSET-SUM $\leq_p$ PARTITION zeigen. Mit anderen Worten wollen wir $A$ bestimmen, so dass
\[ x \in \text{SUBSET-SUM } \Leftrightarrow A(x) \in \text{PARTITION} \]
Die Idee dabei ist, auszunutzen, dass wenn \( x \in \) SUBSET-SUM, folgt, dass es $x$ die Kodierung einer Menge $S$ von natürlichen Zahlen und einer natürlichen Zahl ist, wobei eine Untermenge $U$ von $S$ existiert, deren Elemente zu $t$ summieren. Wir bemerken also, dass dies nicht allzuweit von der Bedingung, die von PARTITION gestellt wird entfernt ist. Wir müssen also sicherstellen, dass es einen ``Gegenpart'' zu $U$ gibt, der ebenfalls zu $t$ aufsummiert, jedoch kein Element aus $U$ enthalten kann. \\
Im Fall \( x \notin \) SUBSET-SUM wollen wir einfach eine Garantie haben, dass die Kodierung des Sets, welches wir an PARTITION weitergeben, keine Partition erlauben kann. \\
Dementsprechend definieren wir A wie folgt: 
\begin{enumerate}[•]
\item Simuliere das $\text{SUBSET-SUM}$ Problem.
\item Falls \(  x \in \text{SUBSET-SUM} \)
	\begin{enumerate}[1.]
	\item $U$ und $t$ werden bei der Simulation gespeichert (z.B. auf einem dafür destinierten Arbeitsband). 
	\item Falls $|U| = 1 \vee (|U|=2 \wedge 0 \in U),\ U' := {t-1,1}$
	\item Sonst $U'=\{t\}$
	\item Übergebe $U \cup U'$ ,d.h. $A(X) = Kodierung(U \cup U') $
	\end{enumerate}
\item Falls \(  x \notin \text{SUBSET-SUM} \) \\
	Übergebe $\{1\}$, d.h. $A(X) = Kodierung( \{ 1 \}) $
\end{enumerate}
\end{homeworkProblem}

%----------------------------------------------------------------------------------------
%	PROBLEM 29
%----------------------------------------------------------------------------------------

\begin{homeworkProblem}
Wir wissen, dass \(G\) regulär und normiert ist. Somit gilt \[ \forall (\alpha,\beta) \in P (\alpha \in \Sigma_N \wedge \beta \in (\Sigma_T \cup \Sigma_T \times \Sigma_N)) \vee (\alpha = S \wedge \beta = \lambda)\] 
%you might as well check this
Offensichtlich folgt unter anderem, dass \(G\) kontextfrei ist. \\
Um eine Grammatik \(G' = ({\Sigma_N}', {\Sigma_T}', P', S) \) für die Sprache \(L' = \{vwv^R | v,w \in L\} \) zu ertsellen, teilen wir die Konstruktion in 3 Teile auf.
\begin{enumerate}[-]

\item Konstruktion von \(v \in L\) \\
Da wir wissen, dass die Regeln, Terminale und Nicht-Terminale aus G kontextfrei sind, können wir sie 	grundsätzlich übernehmen. Dabei ersetzen wir alle Regeln der Form \( X \in \Sigma_N , u \in \Sigma_T \cup \{ \lambda \}, X \rightarrow u  \) durch \(X \rightarrow uH_0\) mit \( H_0 \notin \Sigma_N, H_0 \in {\Sigma_N}' \). \\
Nun müsen wir noch sicherstellen, dass $v$ "gemerkt" wird, um nachher $v^R$ konstruieren zu können. Wir definieren \( | \{N_a, N_b, ... , \} | = | \Sigma_T | mit \{N_a, N_b, ... , \} \cap \Sigma_N = \emptyset \text{ und }  a,b,... \in \Sigma_T\). Jede Regel der Form \( X \in \Sigma_N , u \in \Sigma_T, Y \in \Sigma_N \cup \{\lambda\}, X \rightarrow uY  \) ersetzen wir durch \(X \rightarrow uYN_u\).

\item Konstruktion von \(w \in L\) \\
Grundsätzlich übernehmen wir wieder alle Regeln, Terminale und Nicht-Terminale aus G. Da wir jedoch $w$ von $v$ trennen wollen, bennen wir alle Buchstaben in den Alphabeten und den Regeln um: \(a \in \Sigma_N \cup \Sigma_T \implies a' \in (\Sigma_N)' \). Des weiteren ersetzen wir alle Regeln der Form \( X \in \Sigma_N , u \in \Sigma_T \cup \{ \lambda \}, X \rightarrow u  \) durch \(X \rightarrow uH_1\) mit \( H_1 \notin \Sigma_N, H_1 \in (\Sigma_N)' \). Jedes Vorkommnis von $S$ wird durch $H_0$ ersetzt.

\item Konstruktion von \(v^R\) \\
\( \forall a \in \Sigma_T, (N_a,a) \in P' ; (H_1,\lambda) \in P'\). \\
Mit Hilfe von diesen Regeln kann $v^R$ konstruiert werden, da bei der Konstruktion von $v$ die Platzhalter $N_a$ auf einen Stack oberhalb der Nicht-Literale $H_0$  und $H_1$ gepushed, und in diesem Schritt wieder gepopped werden.
\end{enumerate}

Alle neu erstellen oder modifizierten Regeln sind offensichtlich von der Form \( \alpha \in {\Sigma_N}', \beta \in (\Sigma_N \cup \Sigma_T)^*, \alpha \rightarrow \beta \) und somit kontextfrei.
\end{homeworkProblem}

%---------------------------------------------------------------------------------------- 
\end{document}
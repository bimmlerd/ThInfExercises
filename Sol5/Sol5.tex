%%%%%%%%%%%%%%%%%%%%%%%%%%%%%%%%%%%%%%%%%
% Programming/Coding Assignment
% LaTeX Template
%
% Original author:
% Ted Pavlic (http://www.tedpavlic.com)
%
%
% This template uses a Perl script as an example snippet of code, most other
% languages are also usable. Configure them in the "CODE INCLUSION 
% CONFIGURATION" section.
%
%%%%%%%%%%%%%%%%%%%%%%%%%%%%%%%%%%%%%%%%%

%----------------------------------------------------------------------------------------
%	PACKAGES AND OTHER DOCUMENT CONFIGURATIONS
%----------------------------------------------------------------------------------------

\documentclass{article}
\usepackage[utf8]{inputenc}

\usepackage[swissgerman]{babel}
\usepackage{amsmath}
\usepackage{amsfonts}
\usepackage{fancyhdr} % Required for custom headers
\usepackage{lastpage} % Required to determine the last page for the footer
\usepackage{extramarks} % Required for headers and footers
\usepackage[usenames,dvipsnames]{color} % Required for custom colors
\usepackage{graphicx} % Required to insert images
\usepackage{listings} % Required for insertion of code
\usepackage{courier} % Required for the courier font
\usepackage{enumerate} % used for enumerate args
\usepackage{multicol} % columns

\usepackage{pgf} 
\usepackage{tikz}
\usepackage{forest} % treees :D
\usetikzlibrary{arrows,automata} %for FSM

% Custom commands
\DeclareMathOperator{\Kl}{Kl} %Klassen von Zuständen
\DeclareMathOperator{\Kod}{Kod} %Kodierung von TMs

\usepackage{mathtools}
\DeclarePairedDelimiter{\ceil}{\lceil}{\rceil}
% Shamelessly copied from http://tex.stackexchange.com/questions/43008/absolute-value-symbols
\DeclarePairedDelimiter\abs{\lvert}{\rvert} % nice |x|
\DeclarePairedDelimiter\norm{\lVert}{\rVert} % nice ||x||
% Swap the definition of \abs* and \norm*, so that \abs
% and \norm resizes the size of the brackets, and the 
% starred version does not.
\makeatletter
\let\oldabs\abs
\def\abs{\@ifstar{\oldabs}{\oldabs*}}
\let\oldnorm\norm
\def\norm{\@ifstar{\oldnorm}{\oldnorm*}}
\makeatother


% Margins
\topmargin=-0.45in
\evensidemargin=0in
\oddsidemargin=0in
\textwidth=6.5in
\textheight=9.0in
\headsep=0.25in

\linespread{1.1} % Line spacing

% Set up the header and footer
\pagestyle{fancy}
\lhead{\hmwkAuthorName} % Top left header
%\chead{\hmwkClass\ (\hmwkClassInstructor\): \hmwkTitle} % Top center head
%\rhead{\firstxmark} % Top right header
\rhead{}
\lfoot{\lastxmark} % Bottom left footer
\cfoot{} % Bottom center footer
\rfoot{Seite\ \thepage\ von\ \protect\pageref{LastPage}} % Bottom right footer
\renewcommand\headrulewidth{0.4pt} % Size of the header rule
\renewcommand\footrulewidth{0.4pt} % Size of the footer rule

\setlength\parindent{0pt} % Removes all indentation from paragraphs

%----------------------------------------------------------------------------------------
%	CODE INCLUSION CONFIGURATION
%----------------------------------------------------------------------------------------

\definecolor{MyDarkGreen}{rgb}{0.0,0.4,0.0} % This is the color used for comments
\lstloadlanguages{Pascal} % Load Pascal syntax for listings, for a list of other languages supported see: ftp://ftp.tex.ac.uk/tex-archive/macros/latex/contrib/listings/listings.pdf
\lstset{language=Perl, % Use Pascal in this example
        frame=single, % Single frame around code
        basicstyle=\small\ttfamily, % Use small true type font
        keywordstyle=[1]\color{Blue}\bf, % Pascal functions bold and blue
        keywordstyle=[2]\color{Purple}, % Pascal function arguments purple
        keywordstyle=[3]\color{Blue}\underbar, % Custom functions underlined and blue
        identifierstyle=, % Nothing special about identifiers                                         
        commentstyle=\usefont{T1}{pcr}{m}{sl}\color{MyDarkGreen}\small, % Comments small dark green courier font
        stringstyle=\color{Purple}, % Strings are purple
        showstringspaces=false, % Don't put marks in string spaces
        tabsize=5, % 5 spaces per tab
        %
        % Put standard Pascal functions not included in the default language here
        morekeywords={rand},
        %
        % Put Pascal function parameters here
        morekeywords=[2]{on, off, interp},
        %
        % Put user defined functions here
        morekeywords=[3]{test},
        %
        morecomment=[l][\color{Blue}]{...}, % Line continuation (...) like blue comment
        numbers=left, % Line numbers on left
        firstnumber=1, % Line numbers start with line 1
        numberstyle=\tiny\color{Blue}, % Line numbers are blue and small
        stepnumber=5 % Line numbers go in steps of 5
}

% Creates a new command to include a perl script, the first parameter is the filename of the script (without .p), the second parameter is the caption
\newcommand{\pascalscript}[2]{
\begin{itemize}
\item[]\lstinputlisting[caption=#2,label=#1]{#1.p}
\end{itemize}
}

%----------------------------------------------------------------------------------------
%	DOCUMENT STRUCTURE COMMANDS
%	Skip this unless you know what you're doing
%----------------------------------------------------------------------------------------

% Header and footer for when a page split occurs within a problem environment
%\newcommand{\enterProblemHeader}[1]{
%\nobreak\extramarks{#1}{#1 continued on next page\ldots}\nobreak
%\nobreak\extramarks{#1 (continued)}{#1 continued on next page\ldots}\nobreak
%}

% Header and footer for when a page split occurs between problem environments
%\newcommand{\exitProblemHeader}[1]{
%\nobreak\extramarks{#1 (continued)}{#1 continued on next page\ldots}\nobreak
%\nobreak\extramarks{#1}{}\nobreak
%}

\setcounter{secnumdepth}{0} % Removes default section numbers
\newcounter{homeworkProblemCounter} % Creates a counter to keep track of the number of problems

\newcommand{\homeworkProblemName}{}
\newenvironment{homeworkProblem}[1][Aufgabe \arabic{homeworkProblemCounter}]{ % Makes a new environment called homeworkProblem which takes 1 argument (custom name) but the default is "Problem #"
\stepcounter{homeworkProblemCounter} % Increase counter for number of problems
\renewcommand{\homeworkProblemName}{#1} % Assign \homeworkProblemName the name of the problem
\section{\homeworkProblemName} % Make a section in the document with the custom problem count
%\enterProblemHeader{\homeworkProblemName} % Header and footer within the environment
}{
%\exitProblemHeader{\homeworkProblemName} % Header and footer after the environment
}

\newcommand{\problemAnswer}[1]{ % Defines the problem answer command with the content as the only argument
\noindent\framebox[\columnwidth][c]{\begin{minipage}{0.98\columnwidth}#1\end{minipage}} % Makes the box around the problem answer and puts the content inside
}

\newcommand{\homeworkSectionName}{}
\newenvironment{homeworkSection}[1]{ % New environment for sections within homework problems, takes 1 argument - the name of the section
\renewcommand{\homeworkSectionName}{#1} % Assign \homeworkSectionName to the name of the section from the environment argument
\subsection{\homeworkSectionName} % Make a subsection with the custom name of the subsection
%\enterProblemHeader{\homeworkProblemName\ [\homeworkSectionName]} % Header and footer within the environment
}{
%\enterProblemHeader{\homeworkProblemName} % Header and footer after the environment
}

%----------------------------------------------------------------------------------------
%	NAME AND CLASS SECTION
%----------------------------------------------------------------------------------------

\newcommand{\hmwkTitle}{Übungsaufgaben\ -\ Blatt\ 5} % Assignment title
\newcommand{\hmwkDueDate}{31.\ Oktober\ 2014} % Due date
\newcommand{\hmwkClass}{Theoretische Infomatik} % Course/class
\newcommand{\hmwkClassInstructor}{Prof. Hromkovič} % Teacher/lecturer
\newcommand{\hmwkAuthorName}{Kevin Klein, David Bimmler und Vincent von Rotz} % Your name

%----------------------------------------------------------------------------------------
%	TITLE PAGE
%----------------------------------------------------------------------------------------

\title{
\vspace{2in}
\textmd{\textbf{\hmwkClass:\ \hmwkTitle}}\\
\normalsize\vspace{0.1in}\small{Abgabe\ bis\ \hmwkDueDate}\\
\vspace{0.1in}\large{\textit{\hmwkClassInstructor}
\vspace{3in}
}}
\author{\textbf{\hmwkAuthorName}}
\date{} % Insert date here if you want it to appear below your name

%----------------------------------------------------------------------------------------

\begin{document}

\maketitle

%----------------------------------------------------------------------------------------
%	TABLE OF CONTENTS
%----------------------------------------------------------------------------------------

%\setcounter{tocdepth}{1} % Uncomment this line if you don't want subsections listed in the ToC

\addtocounter{homeworkProblemCounter}{13}
\newpage
%\tableofcontents
%\newpage

%----------------------------------------------------------------------------------------
%	PROBLEM 14
%----------------------------------------------------------------------------------------

\begin{homeworkProblem}
%Wir führen einen Widerspruchsbeweis:
%\\Wir nehmen an, dass $L_1 \in \mathcal{L}_R$
%\\ $\Rightarrow L_1 = L(M)$ für eine TM $M$
%\\ $\Rightarrow \exists k \in \N$


Wir beweisen $L_{1} \not\in \mathcal{L}_{RE}$ indirekt mit einem Widerspruchsbeweis. Dann ist $L_{1} = L(M)$ für eine TM $M$. Weil $M$ eine der Turingmaschinen in der Nummerierung aller Turingmaschinen sein muss, existiert ein $i \in \mathbb{N} - \{0\}$, so dass $M = M_i$. Aber $L_{1}$ kann nicht gleich $L(M_i)$ sein, weil folgende Äquivalenz gilt:
\begin{equation*}
{w_{{i^3}+5}} \in L_{1} \Longleftrightarrow d_{i,{i^3}+5} = 0 \Longleftrightarrow w_{i^3+5} \not\in L(M_i),
\end{equation*}
d.h., $w_{{i^3}+5}$ ist in genau einer der Sprachen $L_{1}$ oder $L(M_i)$.
\\
\\
Dieselbe Methode funktioniert nicht für $L_2$, da wir verwendet haben, dass $L_1$ von irgendeiner TM erzeugt werden muss, falls $L_1 \in \mathcal{L}_{RE}$. D.h., wir haben alle TM $M$ betrachtet, was die Definition von $L_2$ nicht ermöglicht.

\end{homeworkProblem}
%----------------------------------------------------------------------------------------
%	PROBLEM 15
%----------------------------------------------------------------------------------------

\begin{homeworkProblem}
\begin{enumerate}[a)]

\item Wir wissen (aus einem Lemma im Skript), dass $L_U \in \mathcal{L}_{RE}$. Aus Teilaufgauben b) und c) lernen wir, dass $L_U \leq_{EE} L_{union}$ und $L_{union} \leq_{EE} L_U$. Es liegt somit auf der Hand, dass auch $L_{union} \in \mathcal{L}_{RE}$.

\item Wir beweisen $L_U \,{\leq}_{EE}\, L_{union}$ im Formalismus der Turingmaschinen. Wir beschreiben eine TM $M$, die $L_U$ auf $L_{union}$ reduziert. Für die Eingabe $x$ arbeitet $M$ wie folgt:
\par $M$ überprüft, ob die Eingabe die Form $x = \Kod(M_1)\#w$ für eine TM $M_1$ und ein $w \in ({\Sigma}_{bool})^* $ hat.
{\begin{enumerate}[i)]

\item Falls x nicht besagte Form hat, so wird die Kodierung $\Kod(M_{\emptyset})$ einer TM $M_{\emptyset}$ generiert, die keine Eingabe akzeptiert, sondern immer direkt ablehnt. Dann wird noch das Symbol $\#$ angefügt, die $\Kod(M_{\emptyset})$ wiederholt und nochmal ein $\#$ angehängt. Dann hält M mit dem Bandinhalt $ M(x) = \Kod(M_{\emptyset})\#\Kod(M_{\emptyset})\# $.
\item Falls $x=\Kod(M_1)\#w$, dann modifiziert M die Eingabe folgendermassen: Es wird vor die Kodierung der Touringmaschine die Kodierung $\Kod(M_{\emptyset})$ einer zweiten Turingmaschine eingefügt. Diese Turingmaschine akzeptiert wiederum keine Eingabe und lehnt sofort ab. M beendet seine Arbeit mit dem Bandinhalt $M(w) = \Kod(M_{\emptyset})\#\Kod(M_1)\#w$.
\end{enumerate}}

Nun beweisen wir für alle $x \in {\{0,1,\#\}}^*$, dass
\begin{equation*}
x \in L_U \Longleftrightarrow M(x) \in L_{union}
\end{equation*}
Sei $x \in L_U$. Daher ist $x=\Kod(M_1)\#w$ für eine TM $M_1$ und ein Wort $w \in {\{0,1\}}^*$.  Dann ist $M(x) = \Kod(M_{\emptyset})\#\Kod(M_1)\#w$.Wenn $L_U$ nun $x$ akzeptiert, bedeutet das, dass $w \in L(M_1)$. Somit gilt auch $w \in L(M_1) \cup L(M_{\emptyset})$ und also $x \in L_{union}$.
\par Sei $x \not\in L_U$. Das bedeutet entweder, dass es nicht von der Form $x = \Kod(M_1)\#w$ ist, oder dass $M_1 x$ nicht akzeptiert. Im ersten Fall gilt $M(x) = \Kod(M_{\emptyset})\#\Kod(M_{\emptyset})\#$. Da $\lambda \not\in L(M_{\emptyset}) \cup L(M_{\emptyset})$ ist $x \not\in L_{union}$. Ansonsten ist $M(x)=\Kod(M_{\emptyset})\#\Kod(M_1)\#w$. Es gilt: 
\par $x \not\in L_U \Rightarrow w \not\in L(M_1) \Rightarrow w \not\in L(M_{\emptyset}) \cup L(M_1) \Rightarrow x\not\in L_{union}$

\pagebreak
\item Wir beschreiben wiederum eine TM $B$. $B$ arbeitet für die Eingabe x wie folgt:
\par Ist x nicht von der Form $Kod(M)\#Kod(M')\#w$, wobei $M, M'$ Turingmaschinen sind und $w \in ({\Sigma}_{bool})^*$, so gibt $B \: \lambda$ zurück.
\par Ansonsten kreieren wir eine NTM (die ja gleichwertig ist zu einer TM) $M_U$, die zu Beginn nichtdeterministisch entscheidet, ob sie ihre Eingabe mit $M$ oder mit $M'$ überprüfen soll. Wenn einer der beiden Berechnungspfade in einem akzeptierten Zustand landet, so soll $M_U$ akzeptieren. Die jeweligen Verwerfzustände von $M$ und $M'$ werden aufgehoben. Somit terminiert $M_U$ akzeptierend, wenn die Eingabe $\in L(M) \cup L(M')$ ist, und terminiert sonst nicht. $B$ gibt die Kodierung dieser Turingmaschine zurück: $B(x)=Kod(M_U)$

Nun beweisen wir für alle $x \in {\{0,1,\#\}}^*$, dass
\begin{equation*}
x \in L_{union} \Longleftrightarrow B(x) \in L_U
\end{equation*}
Sei $x \in L_{union}$. Dies bedeutet, dass $x = Kod(M)\#Kod(M')\#w$ und $w \in L(M)$ oder $w \in L(M')$. Folglich gilt auch $w \in L(M) \cup L(M')$ und also auch $B(x) \in L_U$.
\par Sei $x \not\in L_{union}$. Dann gilt entweder $x \neq Kod(M)\#Kod(M')\#w$ oder $w \not\in L(M) \wedge w \not\in L(M')$. Im ersten Fall ist $B(x) = \lambda \not\in L_U$. Ansonsten gilt $B(x)=M_U\#w \not\in L_U$, da $M_U$ mit der Eingabe $w$ niemals terminiert.
\end{enumerate}
\end{homeworkProblem}

%----------------------------------------------------------------------------------------
%	PROBLEM 16
%----------------------------------------------------------------------------------------
\begin{homeworkProblem}
\begin{enumerate}[a)]
\item Wir zeigen, dass $L_H\, {\leq}_{EE}\, L_U$, woraus $L_H \, {\leq}_R \, L_U$ folgt. Wir konstruieren also eine Turingmaschine A, die die Eingabe.

Falls x nicht von der Form $Kod(M_3)\#w$ ist, so gibt A $\lambda$ aus.

Falls x aber diese Form hat, so kreieren wir eine neue Turingmaschine $M_3'$, die zu $M_3$ identisch ist bis auf eine Änderung: wir passen $\delta$ an, indem wir jegliche Transition der Form $\delta(q,a) = q_{reject}$ zu $\delta '(q,a) = q_{accept}$ ändern. A gibt dann $Kod(M_3')\#w$ aus.

Nun beweisen wir für alle $x \in {\{0,1,\#\}}^*$, dass
\begin{equation*}
x \in L_H \Longleftrightarrow A(x) \in L_{U}
\end{equation*}
Sei $x \in L_H$. Somit hat x die Form $x=Kod(M_3)\#w$. Nun ist $A(x)=Kod(M_3')\#w$. Wir wissen, dass $M_3$ mit $w$ als Eingabe hält. Zudem wissen wir, dass in $M_3' \: q_{reject}$ durch $q_{accept}$ ersetzt wurde. Dies bedeutet, dass die $M_3'$ akzeptiert, solange sie hält. Somit ist klar, dass $M_3' \: w$ akzeptiert, und somit gilt $A(x) \in L_U$.
\par Sei $x \not\in L_H$. Falls $x$ nicht von der angegebenen Form ist, so ist $A(x)=\lambda$. $\lambda \not\in L_U$ ist offensichtlich. Ansonsten ist $A(x) = Kod(M_3')\#w$. Falls $x \not\in L_H$ bedeutet das, dass die $L_H$ mit der Eingabe $x$ nicht hält. Somit hält auch $A(X)$ mit derselben Eingabe nicht und $A(x) \not\in L_U$.

\item

\end{enumerate}
\end{homeworkProblem}

%----------------------------------------------------------------------------------------

\end{document}